\documentclass{article}
\usepackage[utf8]{inputenc}
\usepackage{fullpage}

\title{Basic Writing Assistant Tool}
\author{Patrick Rock, Ben Sitz, Victoria McCloud}
\date{October 30, 2014}

\begin{document}

\maketitle

\section{What is exactly the function of your tool? That is, what will it do?}
\paragraph{}
Our tool makes writing short summary paragraphs easy. The tool helps you by providing a seed paragraph from Wikipedia about the relevant topic.
As the user makes changes to the seed paragraph, the similarity to the original is calculated by the tool. The user can continue to make the 
document their own until it is sufficently unique.

\section{Why would we need such a tool and who would you expect to use it and benefit from it?}
Summarizing topical information is a common need for students and professionals. 
\begin{itemize}
\item Generate synonyms of words
\item Resturcture sentences, anastrophe
\item Identify similar sentences and suggest a merge
\item Comupte similarity between edited and original document
\end{itemize}

\section{Does this kind of tools already exist? If similar tools exist, how is your tool different from them? Would people care about the difference? How hard is it to build such a tool? What is the challenge?}
There are plagarism detectors available online. These tools are similar to ours, but have a different motiviation. Our tool is designed to help writers
personalize content that they have found online to their style. 

\section{How do you plan to build it? You should mention the data you will use and the core algorithm that you will implement.}

\section{What existing resources can you use?}
We will crawl Wikipedia.

\section{How will you demonstrate the usefulness of your tool?}


\end{document}
