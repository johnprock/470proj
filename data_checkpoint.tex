\documentclass{article}
\usepackage[utf8]{inputenc}
\usepackage{fullpage}
\usepackage{hyperref}



\title{Data Checkpoint}
\author{Patrick Rock, Ben Sitz, Victoria McCloud}
\date{November 6, 2014}

\begin{document}

\maketitle

\section{Introduction}
For our data checkpoint, we pulled pages from wikipedia, extracted the summary paragraph, and implemented a page comparison
function. We also created a basic flask app to embed our functionality into in preparation for the GUI checkpoint.

\section{Wikipedia}
We used the built in python wikipedia library. Getting a page is a simple as one function call! 

\begin{verbatim}
def get_page(t1):
    t1_page = wikipedia.page(t1)
    t1_content = [] 
    t1_content = t1_page.content.split()
    t1_final = []

    x = 0
    while t1_content[x] != '==':
        t1_final.append(t1_content[x]) 
        x+=1

    return str(t1_final)
\end{verbatim}

\section{Similarity}
We use the vector space model and cosine to compute the similarity between two wikipedia documents. 

\begin{verbatim}
def compute_sim(t1, t2): #takes two bodies of strings

    c1 = Counter(t1.split())
    c2 = Counter(t2.split())

    terms = set(c1).union(c2)
    dot_product = sum(c1.get(x, 0) * c2.get(x, 0) for x in terms)
    magnitude1 = math.sqrt(sum(c1.get(x,0)**2 for x in terms))
    magnitude2 = math.sqrt(sum(c2.get(x,0)**2 for x in terms))
    mag_product = magnitude1*magnitude2
    if mag_product == 0:
        return 0 
    return dot_product/mag_product
\end{verbatim}

\section{Flask}
Flask is a python web microframework. We embedded out data functionality into a simple flask app for this checkpoint.

\begin{verbatim}
@app.route("/")
def data_test():
    return render_template('data_test.html')

@app.route("/sim_score.html")
def sim_score():
    t1 = request.args.get('t1')
    t2 = request.args.get('t2')

    sim = compute_sim(t1, t2)

    return render_template('sim_score.html', sim=str(sim))
\end{verbatim}

\end{document}
